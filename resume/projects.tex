%-------------------------------------------------------------------------------
%	SECTION TITLE
%-------------------------------------------------------------------------------
\cvsection{Projects}


%-------------------------------------------------------------------------------
%	CONTENT
%-------------------------------------------------------------------------------
\begin{cventries}

%---------------------------------------------------------
  \cventry
    {Final Year Project} % Role
    {Aerial Eyes – Quadcopter implementing Panorama Stitching and 3D Object Reconstruction} % Event
    {} % Location
    {} % Date(s)
    {
      \newline
      Final Year Project involving the development of a semi-autonomous quadcopter drone, equipped with a Raspberry Pi, running OpenCV for performing Panorama Stitching as well as generating a 3D point cloud using 3D Object Reconstruction using images captured through the PiCam. The drone uses a Python PID controller with​ a Kalman​ filter​ on​ the​ Raspberry​ Pi​ for​ balancing​ with​ inputs​ received​ through​ Wi-Fi​ from​ an​ Android​ App. 
      Delivered a Poster Presentation for the project at Technology Innovation Projects, IST \& CSE Department, CEG.
      Language​ Used​ : Python,​ OpenCV
      Hardware​ Used​ : Raspberry​ Pi,​ IMU(Inertial​ Measurement​ Unit),​ ESC,​ BLDC​ Motor,​ Pi​ Camera
    }

%---------------------------------------------------------
  \cventry
    {Curiculum Project, Publication} % Role
    {Self-balancing​ Robot} % Event
    {} % Location
    {} % Date(s)
    {
      \newline
      Self-autonomous two wheeled Robot, that balances itself, and avoid obstacles, while being controlled through a mobile app. The end use is a scalable, self-balancing, and a self-autonomous robot, that can reach inaccessible places, operate in inhabitable environments, like high radiation or toxic zones, under heavy debris, and in other hazardous​ zones.
      The paper for this project has been published in the International Journal of control Theory and Applications, Vol.9, No.1 (2016).
      Language​ Used​ : Python
      Hardware​ Used​ : Raspberry​ Pi,​ IR​ Sensors,​ IMU(Inertial​ Measurement​ Unit),​ DC​ Geared​ Motors
    }

%---------------------------------------------------------
  \cventry
    {Curiculum Project, Publication} % Role
    {Human​ Pose​ Estimation​ using​ UTI​ Dataset} % Event
    {} % Location
    {} % Date(s)
    {
      \newline
      A MATLAB based surveillance automation project which extracts frames from the video feed and estimates the poses of humans and extrapolates the activity performed by them (such as kicking and boxing) with the help of   adjacent keyframes. The evaluation dataset by University of Texas, Austin was used. Challenges were Real-time   Tuning​ of​ the​ independent​ modules​ to​ increase​ throughput​ and​ obtaining​ Key​ Frames​ for​ the​ activity​ detection.
      The paper for this project has been published in Australian Journal of Basic and Applied Sciences.
      (http://ajbasweb.com/old/ajbas/2016/Special\%201/60-66.pdf)
      Tools​​ Used​ :  MATLAB2012a,​ C++,​ Filters​ (Kalmann​ \& Complementary).
      Concepts​ Used​ :  Image​ Processing,​ Pixel​ Softening​ \& Grouping,​ GrabCut,​ Upper​ Body​ detection,​ frame​ splitting.
    }

%---------------------------------------------------------
  \cventry
    {Kurukshetra Projects} % Role
    {Kurukshetra​ - Student​ Portal} % Event
    {} % Location
    {} % Date(s)
    {
      \newline
      A portal developed for dual purposes - To post/view events in the college doubling as an easy-to-view page of projects done by students (validated upon completion by professors), with searchable tags. The project involved coding​ and​ developing​ the​ Events,​ Projects,​ Sign​ up​ modules​ with​ AJAX​ implementation​ throughout​ the​ site.
      Language​ Used​ :  PHP
      Database​ Used​ :  MongoDB
    }

%---------------------------------------------------------
  \cventry
    {Department Symposium Project} % Role
    {ALEX​ – Ask​ Learn​ EXplore(Student​ Resource​ Sharing​ Portal)} % Event
    {} % Location
    {} % Date(s)
    {
      \newline
      A file repository portal for students to upload, share resources with peers. Uploaded files are tagged autonomously using text mining software, for easy searchability. Files in the form of images are converted to text using tesseract OCR. Discussion threads, like \& share functionalities are also included for ease-of-use. The project​ also​ featured​ taggable​ user​ posts,​ comments​ with​ like​ \& dislike​ options​ and​ a messaging​ system. 
      Technologies​ used​ :  Ruby​ on​ Rails,​ Tesseract​ OCR
    }
%---------------------------------------------------------
\end{cventries}
